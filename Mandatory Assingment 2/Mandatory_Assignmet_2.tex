\documentclass[12pt,
               a4paper,
               article,
               oneside,
               norsk,oldfontcommands]{memoir}
\usepackage{student-kopi}
% Metadata
\date{\today}
\setmodule{MAT4110: Introduction to Numerical Analysis}
\setterm{Fall, 2023}

%-------------------------------%
% Other details
% TODO: Fill these
%-------------------------------%
\title{Mandatory assingment: 1}
\setmembername{Jonas Semprini Næss}  % Fill group member names

%-------------------------------%
% Add / Delete commands and packages
% TODO: Add / Delete here as you need
%-------------------------------%
\makeatletter
\newcommand*{\rom}[1]{\expandafter\@slowromancap\romannumeral #1@}
\makeatother
%\usepackage[utf8]{inputenc}
\usepackage{setspace}
\usepackage[T1]{fontenc}
\usepackage{titling}% the wheel somebody else kindly made for us earlier
\usepackage{fancyhdr}
\usepackage{fancybox}
\usepackage{epigraph} 
\usepackage{tikz}
\usepackage{bm}
\usepackage{pgfplots}
\pgfplotsset{compat=1.12}
\usepackage{lmodern}
\usepackage{enumitem}
\usepackage{framed}
\usepackage{calc}
\usepackage{caption}
\usepackage{subcaption}
\usepackage{fancyvrb}
\usepackage[scaled]{beramono}
\usepackage[final]{microtype}
\usepackage{amssymb}
\usepackage{mathtools}
\usepackage{amsthm}
\usepackage{thmtools}
\usepackage{babel}
\usepackage{csquotes}
\usepackage{listings}
\usetikzlibrary{calc,intersections,through,backgrounds}
\usepackage{tkz-euclide} 
\lstset{basicstyle = \ttfamily}
\usepackage{float}
\usepackage{textcomp}
\usepackage{siunitx}
\usepackage{xcolor}
\usepackage{graphicx}
\usepackage[colorlinks, allcolors = uiolink]{hyperref}
\usepackage[noabbrev]{cleveref}
\pretolerance = 2000
\tolerance    = 6000
\hbadness     = 6000
\newcounter{probnum}[section]
\newcounter{subprobnum}[probnum] 
\usepackage{dirtytalk}
\usepackage{listings}
\usepackage{xcolor}
\usepackage{caption}
\usepackage[section]{placeins}
\usepackage{varwidth}
\usepackage{optidef}
\definecolor{uiolink}{HTML}{0B5A9D}
\usepackage[linesnumbered,ruled,vlined]{algorithm2e}
\usepackage{commath}
\newtheorem{theorem}{Theorem}[section]
\newtheorem{corollary}{Corollary}[theorem]
\newcommand{\Q}{ \qquad \hfill \blacksquare}
\newcommand\myeq{\stackrel{\mathclap{\normalfont{uif}}}{\sim}}
\let\oldref\ref
\renewcommand{\ref}[1]{(\oldref{#1})}
\newtheorem{lemma}[theorem]{Lemma}
\setlength \epigraphwidth {\linewidth}
\setlength \epigraphrule {0pt}
\AtBeginDocument{\renewcommand {\epigraphflush}{center}}
\renewcommand {\sourceflush} {center}
\parindent 0ex
\renewcommand{\thesection}{\roman{section}} 
\renewcommand{\thesubsection}{\thesection.\roman{subsection}}
\newcommand{\KL}{\mathrm{KL}}
\newcommand{\R}{\mathbb{R}}
\newcommand{\E}{\mathbb{E}}
\newcommand{\T}{\top}
\newcommand{\bl}{\left\{}
\newcommand{\br}{\right\}}
\newcommand{\spaze}{\vspace{4mm}\\}
\newcommand{\N}{\mathbb{N}}
\newcommand{\Rel}{\mathbb{R}}
\newcommand{\expdist}[2]{%
        \normalfont{\textsc{Exp}}(#1, #2)%
    }
\newcommand{\expparam}{\bm \lambda}
\newcommand{\Expparam}{\bm \Lambda}
\newcommand{\natparam}{\bm \eta}
\newcommand{\Natparam}{\bm H}
\newcommand{\sufstat}{\bm u}

\definecolor{codegreen}{rgb}{0,0.6,0}
\definecolor{codegray}{rgb}{0.5,0.5,0.5}
\definecolor{codepurple}{rgb}{0.58,0,0.82}
\definecolor{backcolour}{rgb}{0.95,0.95,0.92}

\lstdefinestyle{mystyle}{
    backgroundcolor=\color{backcolour},   
    commentstyle=\color{codegreen},
    keywordstyle=\color{magenta},
    numberstyle=\tiny\color{codegray},
    stringstyle=\color{codepurple},
    basicstyle=\ttfamily\footnotesize,
    breakatwhitespace=false,         
    breaklines=true,                 
    captionpos=b,                    
    keepspaces=true,                 
    numbers=left,                    
    numbersep=5pt,                  
    showspaces=false,                
    showstringspaces=false,
    showtabs=false,                  
    tabsize=2
}

\lstset{style=mystyle}
% Main document
\begin{document}
\header{}
\section*{\centering Problem 1.}
a.) \spaze
\begin{lstlisting}[language=Python, caption=Python example]

import numpy as np
import pandas as pd
import random

matrix = np.array(
    [
        [5, 1 / np.sqrt(2), -1 / np.sqrt(2)],
        [1 / np.sqrt(2), 5 / 2, 7 / 2],
        [-1 / np.sqrt(2), 7 / 2, 5 / 2],
    ]
)

result = pd.DataFrame(columns=["Approx Eigenvalue", "Actual Eigenvalue", "Error"])

num_iterations = 10
actual_eigenvalues, _ = np.linalg.eig(matrix)
print(actual_eigenvalues)
actual_largest_eigenvalue = max(actual_eigenvalues)

n = matrix.shape[0]
b = np.random.rand(n)

for i in range(num_iterations):
    # Power iteration
    b = np.dot(matrix, b)
    # Normalize the vector
    eigenvalue = np.linalg.norm(b)
    error = abs(eigenvalue - actual_largest_eigenvalue)
    b /= eigenvalue
    result.loc[i] = [eigenvalue, actual_largest_eigenvalue, error]


with pd.option_context(
    "display.max_rows",
    None,
    "display.max_columns",
    None,
    "display.precision",
    7,
):
    print(result)


# def inverse_power_method(A, mu, iter, tol=1e-15):
#     Ashift = A - mu * np.identity(A.shape[0])
#     b = np.zeros((len(A), iter + 1))
#     b[:, 0] = np.random.rand(A.shape[0])
#     print(b, b[0])
#     rn = np.ones((iter + 1,))
#     for k in range(num_iterations):
#         b[:, k] = b[:, k] / np.linalg.norm(b[:, k])
#         b[:, k + 1] = np.linalg.solve(Ashift, b[:, k])
#         rn[k + 1] = np.sum(b[:, k + 1]) / np.sum(b[:, k])
#         if abs(rn[k + 1] - rn[k]) < tol:
#             break
#     if k < iter:
#         rn[k + 2 :] = rn[k + 1]
#     return (
#         1.0 / rn[k + 1] + mu,
#         1.0 / rn + mu,
#         b[:, k + 1] / np.linalg.norm(b[:, k + 1]),
#     )


# lamda, v = np.linalg.eig(matrix)
# order = np.abs(lamda).argsort()
# lamda = lamda[order]
# mu = 2
# lamda_shift, lamda_seq, vpm = inverse_power_method(matrix, mu, iter=num_iterations)

# print(
#     "The eigenvalue closest to {} from the shifted power method is {} (exact is {}, error is {})".format(
#         mu, lamda_shift, lamda[1], abs(lamda_shift - lamda[1])
#     )
# )
\end{lstlisting}
\section*{\centering Problem 2.}
a.) \spaze
b.) \spaze 
c.)\spaze 
\section*{\centering Problem 3.}
a.) \spaze
b.) \spaze 
c.)\spaze 
d.) \spaze
\end{document}